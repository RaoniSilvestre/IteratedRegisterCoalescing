\section{Descrição dos Experimentos Computacionais}

\subsection{Metodologia de testes}

Os experimentos foram feitos a partir de um gerador customizado de instâncias de testes, onde 
é possível simular programas da forma mais fiel possível. Foram gerados 15 instâncias de programas 
com características variadas, e o algoritmo de George-Appel foi executado 3 vezes para cada instância.
Por meio dessa base de dados é possível obter métricas sobre os resultados obtidos, tais como:


\begin{description}
  \item[Tempo de execução:] Medido em milissegundos, representa o tempo de cada execução individual da heuristica com uma certa entrada de dados.
  \item[Quantidade de variáveis:] Representa a quantidade total de variáveis ao final do processo de iterativamente simplificar, coaslescer os nós 
    e reescrever o código.
  \item[Quantidade de variáveis despejadas:] Ao fim da quarta execução, na nossa implementação o algoritmo desiste de encontrar uma solução melhor que reduza a 
    quantidade de nós despejados e finaliza sua execução.
\end{description}


\subsubsection{Simulador de programas}

Para conseguir entradas válidas para o algoritmo, é necessário prover de antemão quais são as variáveis, em que linhas as variáveis são utilizadas e também,
quais são os nós provenientes de operações de moves. Para isso, foi construido um simulador simplificado de programas, ele recebe como entrada parâmetros tais como:


\begin{description}
  \item[ID:] Um identificador para salvar em um arquivo os dados simulados.
  \item[Número de variáveis:] Quantidade de variáveis que serão simuladas no programa.
  \item[Quantidade de moves:] Quantidade de arestas que serão criadas como operações de moves, essas arestas são especiais pois, na prática, significa que o início 
    da vida de uma das variáveis é exatamente no final da vida da outra variável, em uma operação do tipo (x = y). Dessa forma, essa informação deve ser considerada
    ao gerar a lista de linhas de usos de cada variável.
  \item[Limite inferior:] Menor quantidade de linhas de código em que a variável é utilizada.
  \item[Limite superior:] Maior quantidade de linhas de código em que a variável é utilizada.
  \item[Linhas de código:] Quantidade de linhas de código totais no código.
\end{description}

Com isso, foram gerados casos de teste variando principalmente o 
número de variáveis: 10, 100, 200, 500 e 1000. E o limite superior de usos de variável: 5, 15, 30. 
Com cada combinação desses valores sendo uma instância de entrada de dados para o algoritmo. Além disso, foram alteradas a quantidade de moves, mas os resultados
não melhoraram ou pioraram. Além disso, como padrão, foi utilizado 32 como a quantidade de registradores disponíveis para coloração.

\subsubsection{Plataforma de execução}

Todas as execuções foram realizadas em uma única máquina para melhorar a consistência dos dados obtidos. A configuração do ambiente foi:

\begin{itemize}
  \item Hardware:
    \begin{description}
      \item[CPU:] Ryzen 5 5600
      \item[RAM:] 16GB DDR4 2400Mhz
    \end{description}
  \item Software:
    \begin{description}
      \item[Sistema Operacional:] Fedora 41
      \item[Linguagem:] Python 3.13
      \item[Project Manager:] UV(\textcite{uv})
    \end{description}
\end{itemize}


\subsection{Como reproduzir os testes}

Para a reprodução dos experimentos, recomenda-se o uso da ferramenta de gerenciamento 
de pacotes \textbf{UV} (\textcite{uv}) para a criação de um ambiente Python consistente.

O processo consiste em dois passos principais:

\begin{enumerate}
    \item \textbf{Gerar as instâncias de entrada:} Execute o comando abaixo no terminal 
    para criar os cinco arquivos de teste que servirão de entrada para o algoritmo.
    
    \begin{center}
        \texttt{uv run src/generator.py}
    \end{center}

    \item \textbf{Executar o algoritmo:} Após a geração das instâncias, utilize o 
    seguinte comando para executar o algoritmo de coloração de grafos sobre os 
    cinco arquivos gerados.
    
    \begin{center}
        \texttt{uv run src/george-appel.py}
    \end{center}
\end{enumerate}

