\section{Introdução}

Registradores são componentes de memória de uso genérico de uma CPU de um computador.
Sua função é de armazenar dados e intruções que serão processados de imediato. É a tecnologia de maior nível na hierárquia de memória em um computador.

Pela sua quantidade reduzida, uso intermitente e alta importância, sua ociosidade ou desperdício não é desejável.
O uso ótimo destes componentes é portante prioritário. Para isso, pesquisa e técnicas de otimização foram desenvolvidas nesta área.

Entre as diversas abordagens disponíveis, uma delas adotadas fora a de coloração de grafos de interferência. Técnica que consiste
em cada variável em dado programa de computador é representada por um vértice no grafo, e suas arestas representam a coexistência
do tempo de vida desta variável no mesmo instante. Cada cor atribuída ao grafo corresponde ao número de registradores de propósito geral disponíveis.

Inicialmente, em 1981 nos laboratórios da IBM, o cientista Chaitin e seus colegas de pesquisa desenvolveram o primeiro algoritmo otimização
do uso de registradores com coloração de grafos de interferência.

O algoritmo é especificado em quatro etapas, construção do grafo de interferência, armazenamento na memória e coloração, spilling
( armazenamento na ram) e recuperação das variáveis na ram e recosntrução do grafo.

Posteriormente, o cientista \textcite{briggs} e seus colegas da Rice University, interessados no problema decidem refinar o
algoritmo de \textcite{chaitin} e companhia, resultando na aprimoração das heurísticas na etapa de coalescência, simplificação e spilling(derramamento).

Por fim, em 1996, pelo Bell Labs na equipe de George e Appel et al., novamente, a técnica de coloração de grafos de interferência
fora otimizada mais uma vez apartir de Briggs e companhia. Agora, a técnica de coalescência admite uma nova heurística, critérios
mais robustos foram atribuidas como prevenção do pior caso do algoritmo de seus antecessores. Uma nova etapa de congelamento
fora adiciona antes de ocorrer o spilling. Atingindo, assim, o estado da arte.
