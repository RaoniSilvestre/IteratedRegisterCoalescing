\section{Conclusão}

Neste trabalho, investigou-se a alocação de registradores via coloração de grafos, comparando
implementações dos algoritmos de \textcite{chaitin}, \textcite{briggs} e \textcite{irc}.  
Observou-se que Chaitin apresenta implementação mais simples, mas tende a maior número
de spills em relação a abordagens refinadas.  
Briggs introduz heurísticas de coalescing otimista que reduzem moves, porém aumentam
a complexidade de análise.  
George-Appel (Iterated Register Coalescing) demonstrou maior taxa de coalescência e menos spills,
sendo uma melhoria direta em relação do de Briggs.  
Em suma, refinamentos em coalescing e heurísticas de spill podem melhorar significativamente a geração de código
sem comprometer o tempo de compilação.
