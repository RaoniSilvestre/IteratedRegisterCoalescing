\section{Conclusão}

A presente análise empírica do algoritmo de alocação de registradores de George-Appel, notável por sua complexa natureza iterativa, demonstrou que seu desempenho
prático é criticamente ditado pelo \textit{live range} das variáveis. Instâncias com \textit{live ranges} longos resultaram em grafos de interferência densos,
cujo custo computacional foi amplificado não apenas na coloração, mas crucialmente na fase de reescrita do programa (\textit{spill}), que provoca uma explosão
não linear no número de variáveis. Em contrapartida, \textit{live ranges} curtos geraram grafos esparsos, processados com eficiência significativamente maior.

Reconhece-se que o simulador utilizado não modela o princípio da localidade, podendo gerar cenários de teste mais desafiadores que programas reais. Esta
limitação aponta para trabalhos futuros, focados no aprimoramento do gerador de instâncias e na validação da heurística em \textit{benchmarks} de compiladores
estabelecidos. Em suma, o estudo valida que a eficiência do algoritmo de George-Appel está intrinsecamente ligada à topologia do grafo de interferência, um fator
mais determinante que o número bruto de variáveis.
